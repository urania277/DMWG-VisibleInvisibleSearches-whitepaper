\section{Introduction} 
\label{Intro}

During the LHC Run-2, ATLAS and CMS searches for dark matter (DM) using missing transverse energy signals %Uli: removed this for the time being~\cite{Beltran:2010ww,Goodman:2010ku,Bai:2010hh}
have begun to use a common set of simplified models, 
%Giorgio: I have added the following line
%selected 
reviewed by the ATLAS/CMS Dark Matter Forum~(DMF), 
%end
to describe how DM would be produced~\cite{Abercrombie:2015wmb}. The models involve TeV-scale mediating particles that couple to quarks and a Dirac fermion DM candidate. The coupling to DM leads to collision events where a high-energy Standard Model (SM) final state recoils against invisible DM particles. Many types of accompanying SM particles are possible, often arising from initial-state radiation, creating a broad set of possible signals involving missing transverse energy (MET). The coupling to quarks, which permits the LHC to produce the mediating particles, also allows the mediators to decay to jets~\cite{Dreiner:2013vla,Chala:2015ama,Fairbairn:2016iuf} or possible to top-quark pairs~\cite{Chala:2015ama,ATLAS:2016pyq,Bauer:2017ota}. Such events, which lack substantial MET, could be used to fully or partially reconstruct the mass and other properties of the mediators.

The LHC Dark Matter Working Group (WG), established by ATLAS, CMS, and the LHC Physics Centre at CERN (LPCC) as the successor of the ATLAS/CMS %Giorgio: removed Dark Matter Forum~(DMF)
DMF~\cite{Abercrombie:2015wmb}, has recommended a set of standardized plots for comparing MET searches channels that differ in the accompanying SM recoil~\cite{Boveia:2016mrp}. The recommendations include depicting the results of these searches in slices of DM mass versus mediator mass for fixed values of the mediator couplings to DM and SM particles. However %Uli, to encourage experimentation, 
the WG 
%declined to 
did not address how these comparisons could incorporate searches for fully-visible decays of the mediators.

As ATLAS and CMS adopted the recommendations for their Run-2 results, both produced preliminary comparisons between visible-decay and invisible-decay searches, starting %Uli: removed this for the time being {\bf cite theory efforts like arXiv:1503.05916 first? Chala:2015ama} 
with an ATLAS comparison of mono-jet, mono-photon, and di-jet searches in the DM-mediator mass plane for a single choice of couplings~\cite{ATLASsummaryplots}, followed by a  comparison of CMS results~\cite{CMS_SummaryPlots_ICHEP} that were also extrapolated to the DM-nucleon cross section for direct-detection DM searches (see also~\cite{Sirunyan:2016iap} for updated results on di-jet resonances). Both ATLAS and CMS results also depict values of the mass parameters where the simplified model reproduces the observed DM density in the standard thermal relic scenario.

The present document discusses some of what has been learned while preparing the above results and includes additional recommendations, stemming from the discussion at the public meeting of the WG in September 2016. Section~\ref{sec:models} adds couplings to leptons for the $s$-channel vector and axial-vector simplified models and provides additional benchmark coupling scenarios that illustrate the relationships amongst the various visible and invisible mediator searches. Section~\ref{sec:relic} discusses a deficiency in the relic density calculations commonly used for the first Run-2 results~\cite{Boveia:2016mrp,ATLASsummaryplots,CMS_SummaryPlots_ICHEP} and compares a new computation with version 2.0.6 of \maddm with the results of an analytic calculation. 
